\documentclass[11pt,letterpaper]{article}
\usepackage[pdftex]{graphicx}
\usepackage[margin=1in]{geometry}
\usepackage[pdftex,hypertexnames=false,linktocpage=true]{hyperref}
\hypersetup{colorlinks=true,linkcolor=blue,anchorcolor=blue,citecolor=blue,filecolor=blue,urlcolor=blue, bookmarksnumbered=true,pdfview=FitB}

% \usepackage[style=chicago-authordate]{biblatex}
% \addbibresource{references/maize-GO-manuscript.bib}

\title{
  \textbf{GO-MAP} \\
  \Large{ \textbf{G}ene \textbf{O}ntology - \textbf{M}eta \textbf{A}nnotator for \textbf{P}lants} \\
  \large {Version 0.1}
}
\author{Kokulapalan (Gokul) Wimalanathan}


\begin{document}
\DeclareGraphicsExtensions{.jpg,.pdf,.mps,.png}

\maketitle
\newpage


\section{Introduction}

\subsection{What is GAMER-pipeline?}
\underline{G}ene \underline{O}ntology - \underline{M}eta \underline{A}nnotator for \underline{P}lants (\textbf{GO-MAP}) is a pipeline that annotates GO terms to plant protein sequences. The pipeline uses three different approaches to annotate GO terms to plant proteins and uses a mix of custom code and existing software tools to assign GO terms. The pipeline was designed to create a high confidence GO annotation dataset for reference proteomes, and it is recommended that the pipeline in it's current form used for that. The main pipeline has been written using Python and R, and other software tools used will be briefly described in the requirements section. Three main approaches used to annotate GO terms are given below.

\subsection{What methods are used to annotate GO terms?}
\subsubsection{Sequence-similarity based methods}
Sequence-similarity based GO annotations were performed using the reciprocal-best-hit method against two different datasets namely TAIR and UniProt. The NCBI BLASTP tool will be used reciprocally to search for similar sequences between the protein sequences of target species being annotated and other datasets. The results from BLASTP search will be processed using R script to determine the reciprocal-best-hits and assign GO terms from TAIR and UniProt to the target species.

\subsubsection{Domain-presence based methods}
Putative protein domains will be assigned to the protein sequence using InterProScan5 pipeline. InterProScan5 uses a java based pipeline to find protein domain signatures in target sequences and assigns GO terms based on the presence of the domain signatures.

\subsubsection{Mixed-method pipelines or tools}
Three best performing pipelines/tools which have competed in the first iteration of the \href{http://biofunctionprediction.org}{CAFA} competition will be used to assign GO terms to proteins. These tools are \href{http://www.medcomp.medicina.unipd.it/Argot2-5/}{Argot2.5}, \href{http://ekhidna.biocenter.helsinki.fi/pannzer}{PANNZER}, and \href{http://montana.informatics.indiana.edu/fanngo/fanngo.html}{FANN-GO}. Each of the tools have specific requirements, setup instructions and pre-processing steps. These all will be explained in the setup section.

\section{Requirements to be installed}
\label{sec:requirements}

The [DOI release] of the GO-MAP pipeline contains code, software, and data files to run the pipeline. Although, there are some basic requirements which need to be installed. The requirements that have to be installed are listed below.

\subsection{What are the requirements that need to be installed to run GO-MAP?}
\label{subsec:install_req}

\begin{itemize}
 \item Hardware
 \begin{itemize}
     \item Storage
     \begin{itemize}
         \item minimum: 250GB
         \item recommended: $\geq$300GB
     \end{itemize}
     \item Memory
     \begin{itemize}
         \item minimum: 16 GB
         \item recommended: $\geq$32 GB
     \end{itemize}
 \end{itemize}
 \item Software
 \begin{itemize}
     \item OS
     \begin{itemize}
         \item linux
     \end{itemize}
     
     \item Programming Languages
     \begin{itemize}
       \item R v3.4
       \item Python v2
       \item Java v1.8
       \item Perl
     \end{itemize}
    
     \item Software
     \begin{itemize}
       \item MATLAB $\geq$v2016a
       \item MySQL/MariaDB
     \end{itemize}
    
     \item Python Packages
     \begin{itemize}
      \item biopython
      \item numpy
      \item scipy
      \item MySQL-python
     \end{itemize}
    
     \item R packages
     \begin{itemize}
      \item ontologyIndex
      \item data.table
      \item ggplot2
      \item futile.logger
      \item jsonlite
     \end{itemize}
\end{itemize}
\end{itemize}


\subsection{What are the software tools needed to run specific annotation methods?}

\begin{itemize}
    \item Sequence-similarity
        \begin{itemize}
            \item BLASTP
        \end{itemize}
    \item Domain-presence
        \begin{itemize}
            \item InterProScan5
        \end{itemize}
    \item Mixed-method Pipelines
    \begin{itemize}
        \item FANN-GO
        \begin{itemize}
            \item MATLAB$^\dagger$  
            \item BLASTP
        \end{itemize}
        \item PANNZER
        \begin{itemize}
            \item MySQL/MariaDB$^\dagger$
            \item BLASTP
        \end{itemize}
        \item Argot2
        \begin{itemize}
            \item Hmmer
            \item BLASTP
            \item Web browser$^{\dagger\ddagger}$
        \end{itemize}
    \end{itemize}
\end{itemize}

$^\dagger$Part of requirements installed as mentioned in this section\\
$^\ddagger$To submit jobs to batch processing \\

The pipeline file downloaded from CyVerse contains the data files and software tools to run the process on a given protein sequence fasta file. The disk space required for the pipeline is large (\textasciitilde{}160GB) and when it runs it will require close to 300GB of disk space.

\section{Setting Up GO-MAP}
\begin{enumerate}
    \item Install dependencies listed in section 2.1
    \item Create a Python2 virtual environment and activate it \textbf{[optional]}
        \begin{minted}{bash}
virtualenv venv
source venv/bin/activate
        \end{minted}
    \item Install required packages for R and Python
        \begin{itemize}
            \item A shell script \mintinline{bash}{install/install_packages.sh} is provided to make the installation of the packages easy
            \item Run the following command from GAMER-pipeline directory
            \begin{minted}[stripall]{bash}
                bash install/install_packages.sh
            \end{minted}
            
        \end{itemize}
    \item Setup MySQL/MariaDB database for Pannzer
        \begin{itemize}
            \item Create a database named \mintinline{bash}{pannzer}
            \item Create the user \mintinline{bash}{pannzer} and grant all privileges on the database \mintinline{bash}{pannzer}
            \item The password should be \mintinline{bash}{pannzer}
            \item If you decide to change any of this, please update the \mintinline{bash}{config.json} file accordingly. The section to be updated is shown below.
            \begin{minted}{json}
{
    "PANNZER": {
                "conf_dir": "mixed-meth/pannzer/conf", 
                "database": {
                    "SQL_DB_PORT": "3306", 
                    "SQL_DB": "pannzer", 
                    "SQL_DB_PASSWORD": "pannzer", 
                    "SQL_DB_USER": "pannzer", 
                    "SQL_DB_HOST": "ccl-gokul.student.iastate.edu"
                }
}
\end{minted}
        \end{itemize}
\end{enumerate}

\section{Running GO-MAP}

GO-MAP is run in two distinct steps. Step one is to run pipeline1.py script to run Sequence-similarity methods, domain-based methods and mixed-method pipelines. Second step is to run pipleine2.py which combines outputs from different tools and creates the aggregate GO annotation dataset.

\begin{enumerate}
 \item Add the input protein fasta file to \mintinline{bash}{input/raw/}
 \item Update \mintinline{bash}{config.json} file as necessary


\begin{itemize}

 \item Update the \mintinline{bash}{pipeline} section
 \begin{minted}{json}
{
    "pipeline": {
        "log_file": "species.inbred.version.log",
        "config_file": "config.json", 
        "version": "0.1", 
        "name": "GO-MAP", 
        "work_dir": "/current/pipeline/location/GO-MAP"
    }
}
 \end{minted}
 \item
       Update the \mintinline{bash}{input} section

       \begin{itemize}

        \item
              Give the correct input FASTA file name
        \item
              If the fasta contains multiple transcripts per gene then put the
              fasta in the \texttt{input/raw} directory and set the
              \texttt{raw\_fasta} parameter
        \item
              If the fasta file contains only on transcript per gene put it in the
              \texttt{input/filt} directory, and set the \texttt{fasta} parameter
        \item
              Update the species, inbred and version parameters for your species
       \end{itemize}
 \item
       {[}Optional{]} Update the \texttt{seq-sim} section

       \begin{itemize}

        \item
              (All the files should be already processed in this section)
       \end{itemize}
 \item
       {[}Optional{]} Update the \texttt{mix-meth} section

       \begin{itemize}

        \item
              (All the files and fields should be already set, except changes to
              database section for PANNZER )
       \end{itemize}
 \item
       {[}Optional{]} Update \texttt{blast} and \texttt{hmmer} sections

       \begin{itemize}

        \item
              This is to enable the correct number cpu threads for these software
       \end{itemize}
 \item
       All other sections should only be updated if things have been
       drastically changed.
\end{itemize}

 \item
       execute \texttt{python\ pipeline1.py\ config.json}
\end{enumerate}

\begin{itemize}

 \item
       The pipeline will generate a number of intermidiate output files
 \item
       Especially the mixed-method tools will require the input fasta to be
       split into smaller chunks. the chunks will be numbered serially.
       (e.g.~test.1.fa, test.2.fa)
 \item
       Argot 2.5 tool will NOT be executed within the pipeline
\end{itemize}

\begin{enumerate}
 \def\labelenumi{\arabic{enumi}.}
 \setcounter{enumi}{3}

 \item
       Submit the files in \texttt{mixed-meth/argot2.5/blast} and
       \texttt{mixed-meth/argot2.5/hmmer} using correct pairing
 \item
       Extract the Argot2.5 result files for each job, in the
       \texttt{mixed-meth/argot2.5/results} directory and rename with correct
       prefix
\end{enumerate}

\begin{itemize}

 \item
       Argot2.5 names all results as \texttt{argot\_results\_ts0.tsv} so the
       file should be renamed correctly (e.g.~test.1.tsv, test.2.tsv)
 \item
       Please do not leave any other file in the argot2.5 results directory,
       otherwise it will influence certain metrics.
\end{itemize}

\begin{enumerate}
 \def\labelenumi{\arabic{enumi}.}
 \setcounter{enumi}{5}

 \item
       execute \texttt{python\ pipeline2.py\ config.json}
\end{enumerate}

\subsection{What are the outputs of GAMER-pipeline?}

GO annotations from GAMER-pipeline will be presented in Go Annotation
2.0 Format (GAF). All the annotations from different methods will
converted to GAF format files and will be saved in sub folders in the
gaf directory. The sub-directory structure in gaf is as follows -
mixed-method (Raw output from mixed-method piplines) - raw (Raw output
from Sequence-similarity and Domain-presence based methods, mixed-method
output filtered to exclude low quality annotations from mixed-method
pipelines) - uniq (Unique annotations from each tool cleaned by removing
duplicate annotations from the raw annotation files) - non\_red
(Non-redundant annotations filtered by removing ancestral GO terms from
the unique annotation files) - agg (Final aggregate dataset created by
combining annotations from all 6 Non-redundant annotation datasets)

\end{document}
