

\subsection{How to run the
 GAMER-pipeline?}\label{how-to-run-the-gamer-pipeline}

GAMER-pipeline is run in two steps using pipeline1.py and pipleine2.py.
First part of the pipeline runs the Sequence-similarity methods and
domain-based methods, and FANN-GO and PANNZER. It also runs the
pre-processing steps for Argot2.5. Second part of the pipeline processes
results from different methods and compiles the final GO annotation
dataset from all differnt approaches. The main steps are given below.

\begin{enumerate}
 \def\labelenumi{\arabic{enumi}.}

 \item
       Add the protein fasta file to \texttt{input/raw/}
 \item
       Make necessary changes to the config.json file
\end{enumerate}

\begin{itemize}

 \item
       Update the \texttt{work\_dir} in the pipeline section
 \item
       Update the \texttt{input} section

       \begin{itemize}

        \item
              Give the correct input FASTA file name
        \item
              If the fasta contains multiple transcripts per gene then put the
              fasta in the \texttt{input/raw} directory and set the
              \texttt{raw\_fasta} parameter
        \item
              If the fasta file contains only on transcript per gene put it in the
              \texttt{input/filt} directory, and set the \texttt{fasta} parameter
        \item
              Update the species, inbred and version parameters for your species
       \end{itemize}
 \item
       {[}Optional{]} Update the \texttt{seq-sim} section

       \begin{itemize}

        \item
              (All the files should be already processed in this section)
       \end{itemize}
 \item
       {[}Optional{]} Update the \texttt{mix-meth} section

       \begin{itemize}

        \item
              (All the files and fields should be already set, except changes to
              database section for PANNZER )
       \end{itemize}
 \item
       {[}Optional{]} Update \texttt{blast} and \texttt{hmmer} sections

       \begin{itemize}

        \item
              This is to enable the correct number cpu threads for these software
       \end{itemize}
 \item
       All other sections should only be updated if things have been
       drastically changed.
\end{itemize}

\begin{enumerate}
 \def\labelenumi{\arabic{enumi}.}
 \setcounter{enumi}{2}

 \item
       execute \texttt{python\ pipeline1.py\ config.json}
\end{enumerate}

\begin{itemize}

 \item
       The pipeline will generate a number of intermidiate output files
 \item
       Especially the mixed-method tools will require the input fasta to be
       split into smaller chunks. the chunks will be numbered serially.
       (e.g.~test.1.fa, test.2.fa)
 \item
       Argot 2.5 tool will NOT be executed within the pipeline
\end{itemize}

\begin{enumerate}
 \def\labelenumi{\arabic{enumi}.}
 \setcounter{enumi}{3}

 \item
       Submit the files in \texttt{mixed-meth/argot2.5/blast} and
       \texttt{mixed-meth/argot2.5/hmmer} using correct pairing
 \item
       Extract the Argot2.5 result files for each job, in the
       \texttt{mixed-meth/argot2.5/results} directory and rename with correct
       prefix
\end{enumerate}

\begin{itemize}

 \item
       Argot2.5 names all results as \texttt{argot\_results\_ts0.tsv} so the
       file should be renamed correctly (e.g.~test.1.tsv, test.2.tsv)
 \item
       Please do not leave any other file in the argot2.5 results directory,
       otherwise it will influence certain metrics.
\end{itemize}

\begin{enumerate}
 \def\labelenumi{\arabic{enumi}.}
 \setcounter{enumi}{5}

 \item
       execute \texttt{python\ pipeline2.py\ config.json}
\end{enumerate}

\subsection{What are the steps needed to setup the pipeline?}\label{what-are-the-steps-needed-to-setup-the-pipeline}

\begin{enumerate}
 \def\labelenumi{\arabic{enumi}.}

 \item
       Install dependencies
 \item
       Install required packages for R and Python
\end{enumerate}

\begin{itemize}

 \item
       A shell script is provided to make the installation of the packages
       easy.
 \item
       Run \texttt{bash\ install/install\_packages.sh} from GAMER-pipeline
       directory
 \item
       Users with a python2 virtual environment please activate before
       running the script
\end{itemize}

\begin{enumerate}
 \def\labelenumi{\arabic{enumi}.}
 \setcounter{enumi}{2}

 \item
       Setup MySQL database for Pannzer
\end{enumerate}

\begin{itemize}

 \item
       Create a database named pannzer
 \item
       Create a user names pannzer and grant all privileges on the database
       pannzer
 \item
       The password should be \texttt{pannzer}
 \item
       If you decide to change any of this, please update the config.json
       {[}mix-meth.PANNZER.database{]} file accordingly.
\end{itemize}

\subsection{How to run the
 GAMER-pipeline?}\label{how-to-run-the-gamer-pipeline}

GAMER-pipeline is run in two steps using pipeline1.py and pipleine2.py.
First part of the pipeline runs the Sequence-similarity methods and
domain-based methods, and FANN-GO and PANNZER. It also runs the
pre-processing steps for Argot2.5. Second part of the pipeline processes
results from different methods and compiles the final GO annotation
dataset from all differnt approaches. The main steps are given below.

\begin{enumerate}
 \def\labelenumi{\arabic{enumi}.}

 \item
       Add the protein fasta file to \texttt{input/raw/}
 \item
       Make necessary changes to the config.json file
\end{enumerate}

\begin{itemize}

 \item
       Update the \texttt{work\_dir} in the pipeline section
 \item
       Update the \texttt{input} section

       \begin{itemize}

        \item
              Give the correct input FASTA file name
        \item
              If the fasta contains multiple transcripts per gene then put the
              fasta in the \texttt{input/raw} directory and set the
              \texttt{raw\_fasta} parameter
        \item
              If the fasta file contains only on transcript per gene put it in the
              \texttt{input/filt} directory, and set the \texttt{fasta} parameter
        \item
              Update the species, inbred and version parameters for your species
       \end{itemize}
 \item
       {[}Optional{]} Update the \texttt{seq-sim} section

       \begin{itemize}

        \item
              (All the files should be already processed in this section)
       \end{itemize}
 \item
       {[}Optional{]} Update the \texttt{mix-meth} section

       \begin{itemize}

        \item
              (All the files and fields should be already set, except changes to
              database section for PANNZER )
       \end{itemize}
 \item
       {[}Optional{]} Update \texttt{blast} and \texttt{hmmer} sections

       \begin{itemize}

        \item
              This is to enable the correct number cpu threads for these software
       \end{itemize}
 \item
       All other sections should only be updated if things have been
       drastically changed.
\end{itemize}

\begin{enumerate}
 \def\labelenumi{\arabic{enumi}.}
 \setcounter{enumi}{2}

 \item
       execute \texttt{python\ pipeline1.py\ config.json}
\end{enumerate}

\begin{itemize}

 \item
       The pipeline will generate a number of intermidiate output files
 \item
       Especially the mixed-method tools will require the input fasta to be
       split into smaller chunks. the chunks will be numbered serially.
       (e.g.~test.1.fa, test.2.fa)
 \item
       Argot 2.5 tool will NOT be executed within the pipeline
\end{itemize}

\begin{enumerate}
 \def\labelenumi{\arabic{enumi}.}
 \setcounter{enumi}{3}

 \item
       Submit the files in \texttt{mixed-meth/argot2.5/blast} and
       \texttt{mixed-meth/argot2.5/hmmer} using correct pairing
 \item
       Extract the Argot2.5 result files for each job, in the
       \texttt{mixed-meth/argot2.5/results} directory and rename with correct
       prefix
\end{enumerate}

\begin{itemize}

 \item
       Argot2.5 names all results as \texttt{argot\_results\_ts0.tsv} so the
       file should be renamed correctly (e.g.~test.1.tsv, test.2.tsv)
 \item
       Please do not leave any other file in the argot2.5 results directory,
       otherwise it will influence certain metrics.
\end{itemize}

\begin{enumerate}
 \def\labelenumi{\arabic{enumi}.}
 \setcounter{enumi}{5}

 \item
       execute \texttt{python\ pipeline2.py\ config.json}
\end{enumerate}